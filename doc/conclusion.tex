% ---------------------------------*- Latex -*---------------------------------
% Filename: conclusion.tex
% Description: 
% Author: Fabian Wermelinger
% Email: fabianw@student.ethz.ch
% Created: Thu Dec 15 20:23:54 2011 (+0100)
% Version: 
% Last-Updated: Thu Dec 15 20:24:40 2011 (+0100)
%           By: Fabian Wermelinger
%     Update #: 2
% -----------------------------------------------------------------------------
% conclusion.tex starts here
% -----------------------------------------------------------------------------

\section{Conclusion}
\label{sec:conclusion}

All in all our simulations showed some interesting phenomena and sensitivities to different parameters which we are going to summerize in the following. Through the project some interesting new questions arised while others could not be sufficiently be answered. \\
As described in the previous sections the type of the network did not had a to big influence on the results of our simulation. Perhaps were these quite similar results for the different network types a result of the well connected networks we used for most of our simulations. The biggest difference about the random graph and small world network that our simulations showed was that for a random graph the spread of opinions across different clusters was not as predictable as for small world networks. This could be a result that not all the nodes of a random graph have such a short connection as in a small world network. Overall our simulations showed for different parameters that the diffusion of opinions is in small world networks in principal better. \\
Section~\ref{sec:comparisontoreal} showed that there is a relation of the model implemented in \matlab to a real world event.  In order to perform more event specific simulations, parameter sets determined from data of the particular event must be provided.  In general we conclude that the experiment and the model behave closer to each other if the elements in $\mathcal{N}$ and $\mathcal{S}_i$ \emph{increase}.  However, the number of elements in $\mathcal{S}_i$ must still be significantly less than the number of elements in $\mathcal{N}$. \\


%%% Local Variables: 
%%% mode: latex
%%% TeX-master: "master"
%%% End: 

% -----------------------------------------------------------------------------
% conclusion.tex ends here
% -----------------------------------------------------------------------------
