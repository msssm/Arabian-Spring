\section{Discussion of Results}
\label{sec:discussionofresults}

\subsection{Influence of \texttt{noize}}
\label{sec:influencenoize}
As figure \ref{influencenoize} shows, the noize could have quite an influence on the results of the simulation. In our network we have in total 1200 nodes. All the simulations in figure \ref{influencenoize} have the same \texttt{maxUpdate}=0.02. This means that in every time step 24 agents get updated. If the noize is now for example 0.01 only 0.24 agents choose randomly their state. In the case of a noize of 0.1 get 2.4 agents every time step a randomly after a uniform distribution chosen mind state. After this "noize update" the concerned agents are removed from the sequential update list so that they could not be updated twice.\\
As figure \ref{influencenoize} now shows and explained above a noize of 0.01 does not really change the results of the simulation as it has nearly no influence. 

\begin{figure}
\centering
\includegraphics[width=0.25\textwidth]{batchRun__kHalf=2-2-2_maxUpdate=0.02_noize=0_nbrDepth=1/network0-crop.pdf}
\hfill
\includegraphics[width=0.25\textwidth]{batchRun__kHalf=2-2-2_maxUpdate=0.02_noize=0.01_nbrDepth=1/network0-crop.pdf}
\hfill
\includegraphics[width=0.25\textwidth]{batchRun__kHalf=2-2-2_maxUpdate=0.02_noize=0.1_nbrDepth=1/network0-crop.pdf}

\includegraphics[width=0.25\textwidth]{batchRun__kHalf=2-2-2_maxUpdate=0.02_noize=0_nbrDepth=1/network250-crop.pdf}
\hfill
\includegraphics[width=0.25\textwidth]{batchRun__kHalf=2-2-2_maxUpdate=0.02_noize=0.01_nbrDepth=1/network250-crop.pdf}
\hfill
\includegraphics[width=0.25\textwidth]{batchRun__kHalf=2-2-2_maxUpdate=0.02_noize=0.1_nbrDepth=1/network250-crop.pdf}

\includegraphics[width=0.25\textwidth]{batchRun__kHalf=2-2-2_maxUpdate=0.02_noize=0_nbrDepth=1/network500-crop.pdf}
\hfill
\includegraphics[width=0.25\textwidth]{batchRun__kHalf=2-2-2_maxUpdate=0.02_noize=0.01_nbrDepth=1/network500-crop.pdf}
\hfill
\includegraphics[width=0.25\textwidth]{batchRun__kHalf=2-2-2_maxUpdate=0.02_noize=0.1_nbrDepth=1/network500-crop.pdf}


\includegraphics[width=0.25\textwidth]{batchRun__kHalf=2-2-2_maxUpdate=0.02_noize=0_nbrDepth=1/network750-crop.pdf}
\hfill
\includegraphics[width=0.25\textwidth]{batchRun__kHalf=2-2-2_maxUpdate=0.02_noize=0.01_nbrDepth=1/network750-crop.pdf}
\hfill
\includegraphics[width=0.25\textwidth]{batchRun__kHalf=2-2-2_maxUpdate=0.02_noize=0.1_nbrDepth=1/network750-crop.pdf}

\includegraphics[width=0.25\textwidth]{batchRun__kHalf=2-2-2_maxUpdate=0.02_noize=0_nbrDepth=1/network1000-crop.pdf}
\hfill
\includegraphics[width=0.25\textwidth]{batchRun__kHalf=2-2-2_maxUpdate=0.02_noize=0.01_nbrDepth=1/network1000-crop.pdf}
\hfill
\includegraphics[width=0.25\textwidth]{batchRun__kHalf=2-2-2_maxUpdate=0.02_noize=0.1_nbrDepth=1/network1000-crop.pdf}

\caption{The influence of the noize for the three cases with on the left noize = 0, in the middle 0.01 and on the right 0.1. From top to bottom are the different time steps with the beginning and then increasing in 250 time steps till a 100 time steps.}
\label{influencenoize}
\end{figure}

\subsection{Influence of \texttt{maxAgentupdate}}
\label{sec:maxAgentUpdate}

\subsection{Influence of \texttt{nbrDepth} = Neighbor Depth}
\label{sec:nbrDepth}